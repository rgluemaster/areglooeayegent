\documentclass[11pt]{article}

\usepackage[top=1in, bottom=1in, left=1in, right=1in]{geometry}
\usepackage{amsfonts}
\usepackage{graphicx}
\usepackage{float}
\usepackage[utf8]{inputenc}
\usepackage[section]{placeins}
\usepackage{abstract}
\usepackage{amsmath}
\usepackage{amsfonts}
\usepackage{amssymb}
\usepackage{enumitem}
\renewcommand{\abstractnamefont}{\normalfont\bfseries} 
\renewcommand{\abstracttextfont}{\normalfont} 
\numberwithin{equation}{section}
\newcommand{\argmax}{\operatornamewithlimits{argmax}}

\begin{document}

\title{DMUU Project Report}
\author{\begin{tabular}{cc}
Roland Hellström & Sebastian Ånerud \\
XXXXXX-xxxx & 910407-5958 \\
asdf.comasd@ & anerud@student.chalmers.se
\end{tabular}}
\date{\today}
\maketitle

\begin{center}
\centering
\includegraphics[scale=1]{DOGE}
\end{center}

\newpage 

\begin{flushleft}

\section{Introduction}

This project consists of building an agent that is able to act and take optimal decisions in unknown and arbitrary environments. The only thing known about the environments, that the agent is supposed to act in, is the number of states, the number of actions and that the underlying model is a Markov Decision Process. Also it is known that one of the environments is a Partially Observable MDP (POMDP). However, the agent presented in this report does not implement any algorithm specifically designed to act well in such an environment. The algorithms used by the agent are a Generalized Stochastic Value Iteration- and an Upper Confidence Bound-Algorithm. Some of the environments that the agent are tested on are a simple 2-arm bandit-, n-armed bandit-, mines-, tic tac toe- and chain-environment.

\section{Method}

In this section the choice of algorithms and the motivation behind choice will be discussed briefly. However, the algorithms themselves will not be explained thoroughly and it is assumed that the reader either have knowledge about the algorithms or takes the time to understand them before reading any further. Also the environments used for testing and verifying the agent will be discussed.  \newline

\subsection{Testing environments}

\subsubsection{Bandit environments}
\subsubsection{Mines environment}
\subsubsection{Chain environment}
\subsubsection{Tic tac toe environment}

\subsection{Algorithms used by agent}

\subsubsection{Upper Confidence Bound}

\subsubsection{Generalized Stochastic Value iteration}

\section{Results}

Present the results from the different environments.

\section{Discussion}

Discuss the results. bla bla voff voff.

\section{Conclusion}

very much conclude. so amaze. wow.

\end{flushleft}

\end{document}